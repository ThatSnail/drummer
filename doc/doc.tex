\title{Title}
\documentclass[12pt]{article}

\usepackage{amsmath}
\usepackage{ amssymb }
\usepackage{braket}
\usepackage{hyperref}

\begin{document}
\maketitle

We take $J_m$ to be the Bessel function of order $m$.  $m=0$ is spherical, etc.

Consider a drum with radius $a$, and "tension" parameter $c$.  Solution for a circular membrane are:
\begin{align*}
    u_{mn}(r, \theta, t) = R(r) \Theta(\theta) T(t)
\end{align*}
with
\begin{align*}
    R(r) &= J_m( k_{mn} * r) \\
    \Theta(\theta) &= \cos (m(\theta - \theta_0)) \\
    T(t) &= \cos (c k_{mn} (t - t_0))
\end{align*}
$k_mn$ is such that $J_m(k_mn a) = 0$.  They are the normalized zeroes of the Bessel functions. \\
The general solution is a superposition of these:
\begin{align*}
    u(r, \theta, t) = \sum_{m = 0}^\infty \sum_{n=1}^\infty c_{mn} u_{mn}(r, \theta, t)
\end{align*}
for any coefficients $c_{mn}$.  \\
When we hit the drum at position $(r_0, \theta_0)$ at time $t_0$, what should the coefficients be?  We need to choose coefficients such that the state of the drum looks like a Dirac impulse at the hit point.  We use the Hankel transform:
\begin{align*}
    c_n &= \frac{\int\limits_0^a dr f(r) J_m(k_{mn} r) r}{\frac{a^2}{2} J^2_{m+1}(k_{mn} a)} \\
        &= \frac{G J_m(\frac{\alpha_{mn}}{a} r_0)}{\frac{a^2}{2} J^2_{m+1}(k_{mn} a)} \\
\end{align*}
where we plugged in our impulse:
\begin{align*}
    f(r) = G \delta(r - r_0)
\end{align*}
for some strength factor $G$.  See bottom of \url{http://www.hit.ac.il/staff/benzionS/Differential.Equations/Orthogonality_of_Bessel_functions.htm}. \\

We take the $\theta_0$ and $t_0$ in our $\Theta$ and $T$ functions to be where we hit the drum.  These functions are maximal at these points.

\end{document}
